%% Generated by Sphinx.
\def\sphinxdocclass{report}
\documentclass[letterpaper,10pt,italian,openany,oneside]{sphinxmanual}
\ifdefined\pdfpxdimen
   \let\sphinxpxdimen\pdfpxdimen\else\newdimen\sphinxpxdimen
\fi \sphinxpxdimen=.75bp\relax

\PassOptionsToPackage{warn}{textcomp}
\usepackage[utf8]{inputenc}
\ifdefined\DeclareUnicodeCharacter
% support both utf8 and utf8x syntaxes
  \ifdefined\DeclareUnicodeCharacterAsOptional
    \def\sphinxDUC#1{\DeclareUnicodeCharacter{"#1}}
  \else
    \let\sphinxDUC\DeclareUnicodeCharacter
  \fi
  \sphinxDUC{00A0}{\nobreakspace}
  \sphinxDUC{2500}{\sphinxunichar{2500}}
  \sphinxDUC{2502}{\sphinxunichar{2502}}
  \sphinxDUC{2514}{\sphinxunichar{2514}}
  \sphinxDUC{251C}{\sphinxunichar{251C}}
  \sphinxDUC{2572}{\textbackslash}
\fi
\usepackage{cmap}
\usepackage[T1]{fontenc}
\usepackage{amsmath,amssymb,amstext}
\usepackage{babel}



\usepackage{times}
\expandafter\ifx\csname T@LGR\endcsname\relax
\else
% LGR was declared as font encoding
  \substitutefont{LGR}{\rmdefault}{cmr}
  \substitutefont{LGR}{\sfdefault}{cmss}
  \substitutefont{LGR}{\ttdefault}{cmtt}
\fi
\expandafter\ifx\csname T@X2\endcsname\relax
  \expandafter\ifx\csname T@T2A\endcsname\relax
  \else
  % T2A was declared as font encoding
    \substitutefont{T2A}{\rmdefault}{cmr}
    \substitutefont{T2A}{\sfdefault}{cmss}
    \substitutefont{T2A}{\ttdefault}{cmtt}
  \fi
\else
% X2 was declared as font encoding
  \substitutefont{X2}{\rmdefault}{cmr}
  \substitutefont{X2}{\sfdefault}{cmss}
  \substitutefont{X2}{\ttdefault}{cmtt}
\fi


\usepackage[Sonny]{fncychap}
\ChNameVar{\Large\normalfont\sffamily}
\ChTitleVar{\Large\normalfont\sffamily}
\usepackage{sphinx}

\fvset{fontsize=\small}
\usepackage{geometry}

% Include hyperref last.
\usepackage{hyperref}
% Fix anchor placement for figures with captions.
\usepackage{hypcap}% it must be loaded after hyperref.
% Set up styles of URL: it should be placed after hyperref.
\urlstyle{same}

\usepackage{sphinxmessages}
\setcounter{tocdepth}{1}



\title{filestat}
\date{08 giu 2019}
\release{1.0.0}
\author{Francesco Pio Stelluti, Francesco Coppola}
\newcommand{\sphinxlogo}{\vbox{}}
\renewcommand{\releasename}{Release}
\makeindex
\begin{document}

\ifdefined\shorthandoff
  \ifnum\catcode`\=\string=\active\shorthandoff{=}\fi
  \ifnum\catcode`\"=\active\shorthandoff{"}\fi
\fi

\pagestyle{empty}
\sphinxmaketitle
\pagestyle{plain}
\sphinxtableofcontents
\pagestyle{normal}
\phantomsection\label{\detokenize{index::doc}}


All’interno delle seguenti pagine sarà possibile trovare la documentazione
generata per il progetto \sphinxstylestrong{filestat}, utility di sistema in grado di monitorare informazioni
su file e directory, realizzato per il corso di \sphinxstyleemphasis{Laboratorio di Sistemi Operativi}
dell’anno 2018/2019.

Lo sviluppo di tale codice è da attribuire interamente agli studenti \sphinxstylestrong{Francesco Pio Stelluti} e \sphinxstylestrong{Francesco Coppola}.


\chapter{Introduzione}
\label{\detokenize{introduzione:introduzione}}\label{\detokenize{introduzione::doc}}\begin{quote}

\sphinxstyleemphasis{«Software application that is able to monitor a group of file taking information about them»}
\end{quote}

L’utility di sistema realizzata prende il nome di \sphinxstylestrong{filestat} e consente il monitoraggio avanzato di
file e direcotry all’interno del sitema, forenendo dei report e delle statistiche riguardo a quest’utlimi.


\section{Sinossi del programma e avvio}
\label{\detokenize{introduzione:sinossi-del-programma-e-avvio}}
La sinossi del programma è:

\begin{sphinxVerbatim}[commandchars=\\\{\}]
\PYG{n}{filestat} \PYG{p}{[}\PYG{n}{options}\PYG{p}{]} \PYG{p}{[}\PYG{n+nb}{input}\PYG{p}{]} \PYG{p}{[}\PYG{n}{output}\PYG{p}{]}
\end{sphinxVerbatim}

Dove:
\begin{itemize}
\item {} 
\sphinxcode{\sphinxupquote{input}} è il file di input dove vengono definiti i parametri di esecuzione del programma, se omesso viene usato il file \sphinxcode{\sphinxupquote{filestat.in}};

\item {} 
\sphinxcode{\sphinxupquote{output}} è il file di output dove vengono collezionati i dati raccolti, se omesso viene usato il file \sphinxcode{\sphinxupquote{filestat.db}}. Le informazioni presenti nel file di output vengono \sphinxstyleemphasis{aggiornate} ad ogni esecuzione del programma \sphinxstylestrong{(e non soprascritte)}.

\end{itemize}

Le possibili opzioni sono:

\begin{sphinxVerbatim}[commandchars=\\\{\}]
\PYG{o}{\PYGZhy{}}\PYG{o}{\PYGZhy{}}\PYG{n}{verbose}\PYG{o}{\textbar{}}\PYG{o}{\PYGZhy{}}\PYG{n}{v}
\PYG{o}{\PYGZhy{}}\PYG{o}{\PYGZhy{}}\PYG{n}{stat}\PYG{o}{\textbar{}}\PYG{o}{\PYGZhy{}}\PYG{n}{s}
\PYG{o}{\PYGZhy{}}\PYG{o}{\PYGZhy{}}\PYG{n}{report}\PYG{o}{\textbar{}}\PYG{o}{\PYGZhy{}}\PYG{n}{r}
\PYG{o}{\PYGZhy{}}\PYG{o}{\PYGZhy{}}\PYG{n}{history}\PYG{o}{\textbar{}}\PYG{o}{\PYGZhy{}}\PYG{n}{h} \PYG{o}{\PYGZlt{}}\PYG{n}{filepah}\PYG{o}{\PYGZgt{}}
\PYG{o}{\PYGZhy{}}\PYG{o}{\PYGZhy{}}\PYG{n}{user}\PYG{o}{\textbar{}}\PYG{o}{\PYGZhy{}}\PYG{n}{u} \PYG{o}{\PYGZlt{}}\PYG{n}{userId}\PYG{o}{\PYGZgt{}}
\PYG{o}{\PYGZhy{}}\PYG{o}{\PYGZhy{}}\PYG{n}{group}\PYG{o}{\textbar{}}\PYG{o}{\PYGZhy{}}\PYG{n}{g} \PYG{o}{\PYGZlt{}}\PYG{n}{groupId}\PYG{o}{\PYGZgt{}}
\PYG{o}{\PYGZhy{}}\PYG{o}{\PYGZhy{}}\PYG{n}{length}\PYG{o}{\textbar{}}\PYG{o}{\PYGZhy{}}\PYG{n}{l} \PYG{o}{\PYGZlt{}}\PYG{n+nb}{min}\PYG{o}{\PYGZgt{}}\PYG{p}{:}\PYG{o}{\PYGZlt{}}\PYG{n+nb}{max}\PYG{o}{\PYGZgt{}}
\PYG{o}{\PYGZhy{}}\PYG{o}{\PYGZhy{}}\PYG{n}{noscan}
\end{sphinxVerbatim}

La descrizione è la seguente:
\begin{itemize}
\item {} 
\sphinxcode{\sphinxupquote{-{-}verbose\textbar{}-v}}: durante l’esecuzione il programma mostra a video le informazioni sui file elaborati, ed i dati raccolti;

\item {} 
\sphinxcode{\sphinxupquote{-{-}stat\textbar{}-s}}: vengono mostrate sullo standard output le seguenti statistiche:
\begin{itemize}
\item {} 
numero di file monitorati;

\item {} 
numero di link;

\item {} 
numero di directory;

\item {} 
dimensione totale;

\item {} 
dimensione media;

\item {} 
dimensione massima;

\item {} 
dimensione minima (in byte).

\end{itemize}

\item {} 
\sphinxcode{\sphinxupquote{-{-}report\textbar{}-r}}: al termine dell’esecuzione vengono mostrati sullo standard output le informazioni riguardanti numero di file elaborati, tempo di elaborazione, dimensione massima del file;

\item {} 
\sphinxcode{\sphinxupquote{-{-}history\textbar{}-h \textless{}filepah\textgreater{}}}: stampa sullo standard output la cronologia delle informazioni riguardanti il file \sphinxcode{\sphinxupquote{\textless{}filepah\textgreater{}}};

\item {} 
\sphinxcode{\sphinxupquote{-{-}user\textbar{}-u \textless{}userId\textgreater{}}}: stampa sullo standard output le informazioni di tutti i file di proprietà di \sphinxcode{\sphinxupquote{\textless{}userId\textgreater{}}}

\item {} 
\sphinxcode{\sphinxupquote{-{-}group\textbar{}-g \textless{}groupId\textgreater{}}}: stampa sullo standard output le informazioni di tutti i file di proprietà di \sphinxcode{\sphinxupquote{\textless{}groupId\textgreater{}}}

\item {} 
\sphinxcode{\sphinxupquote{-{-}length\textbar{}-l \textless{}min\textgreater{}:\textless{}max\textgreater{}}}: stampa sullo schermo le informazioni di tutti i file di dimensione (in byte) compresa tra \sphinxcode{\sphinxupquote{\textless{}min\textgreater{}}} e \sphinxcode{\sphinxupquote{\textless{}max\textgreater{}}} (\sphinxcode{\sphinxupquote{:\textless{}max\textgreater{}}} indica ogni file di dimensione al più \sphinxcode{\sphinxupquote{\textless{}max\textgreater{}}}, \sphinxcode{\sphinxupquote{\textless{}min\textgreater{}:}} e \sphinxcode{\sphinxupquote{\textless{}min\textgreater{}}} indicano ogni file di dimensione almeno \sphinxcode{\sphinxupquote{\textless{}min\textgreater{}}})

\item {} 
\sphinxcode{\sphinxupquote{-{-}noscan}}: se presente questa opzione non viene effettuata la raccolta dei dati, ma vengono presentati solo le informazioni presenti del file di output.

\end{itemize}


\section{Formato del file di input}
\label{\detokenize{introduzione:formato-del-file-di-input}}
I parametri di esecuzione di un programma vengono definiti in un file di testo costituito da una sequenza di righe della seguente forma:

\begin{sphinxVerbatim}[commandchars=\\\{\}]
\PYG{o}{\PYGZlt{}}\PYG{n}{path}\PYG{o}{\PYGZgt{}} \PYG{p}{[}\PYG{n}{r}\PYG{p}{]} \PYG{p}{[}\PYG{n}{l}\PYG{p}{]}
\end{sphinxVerbatim}

Dove \sphinxcode{\sphinxupquote{r}} indica che occorre leggere ricorsivamente i file nelle directory sottostanti (applicando le stesse opzioni) mentre \sphinxcode{\sphinxupquote{l}} indica che i link devono essere trattati come file/directory regolari, in questo caso le informazioni collezionate fanno riferimento al file riferito dal link e non a link stesso.


\section{Formato del file di output}
\label{\detokenize{introduzione:formato-del-file-di-output}}
I dati raccolti vengono salvati usando il seguente formato:

\begin{sphinxVerbatim}[commandchars=\\\{\}]
\PYG{c+c1}{\PYGZsh{} \PYGZlt{}path1\PYGZgt{}}
\PYG{o}{\PYGZlt{}}\PYG{n}{data1}\PYG{o}{\PYGZgt{}} \PYG{o}{\PYGZlt{}}\PYG{n}{uid1}\PYG{o}{\PYGZgt{}} \PYG{o}{\PYGZlt{}}\PYG{n}{gid1}\PYG{o}{\PYGZgt{}} \PYG{o}{\PYGZlt{}}\PYG{n}{dim1}\PYG{o}{\PYGZgt{}} \PYG{o}{\PYGZlt{}}\PYG{n}{perm1}\PYG{o}{\PYGZgt{}} \PYG{o}{\PYGZlt{}}\PYG{n}{acc1}\PYG{o}{\PYGZgt{}} \PYG{o}{\PYGZlt{}}\PYG{n}{change1}\PYG{o}{\PYGZgt{}} \PYG{o}{\PYGZlt{}}\PYG{n}{mod1}\PYG{o}{\PYGZgt{}} \PYG{o}{\PYGZlt{}}\PYG{n}{nlink1}\PYG{o}{\PYGZgt{}}
\PYG{o}{.}\PYG{o}{.}\PYG{o}{.}
\PYG{o}{\PYGZlt{}}\PYG{n}{data\PYGZus{}n}\PYG{o}{\PYGZgt{}} \PYG{o}{\PYGZlt{}}\PYG{n}{uid\PYGZus{}n}\PYG{o}{\PYGZgt{}} \PYG{o}{\PYGZlt{}}\PYG{n}{gid\PYGZus{}n}\PYG{o}{\PYGZgt{}} \PYG{o}{\PYGZlt{}}\PYG{n}{dim\PYGZus{}n}\PYG{o}{\PYGZgt{}} \PYG{o}{\PYGZlt{}}\PYG{n}{perm\PYGZus{}n}\PYG{o}{\PYGZgt{}} \PYG{o}{\PYGZlt{}}\PYG{n}{acc\PYGZus{}n}\PYG{o}{\PYGZgt{}} \PYG{o}{\PYGZlt{}}\PYG{n}{change\PYGZus{}n}\PYG{o}{\PYGZgt{}} \PYG{o}{\PYGZlt{}}\PYG{n}{mod\PYGZus{}n}\PYG{o}{\PYGZgt{}} \PYG{o}{\PYGZlt{}}\PYG{n}{nlink\PYGZus{}n}\PYG{o}{\PYGZgt{}}
\PYG{c+c1}{\PYGZsh{}\PYGZsh{}\PYGZsh{}}
\PYG{c+c1}{\PYGZsh{} \PYGZlt{}path2\PYGZgt{}}
\PYG{o}{\PYGZlt{}}\PYG{n}{data1}\PYG{o}{\PYGZgt{}} \PYG{o}{\PYGZlt{}}\PYG{n}{uid1}\PYG{o}{\PYGZgt{}} \PYG{o}{\PYGZlt{}}\PYG{n}{gid1}\PYG{o}{\PYGZgt{}} \PYG{o}{\PYGZlt{}}\PYG{n}{dim1}\PYG{o}{\PYGZgt{}} \PYG{o}{\PYGZlt{}}\PYG{n}{perm1}\PYG{o}{\PYGZgt{}} \PYG{o}{\PYGZlt{}}\PYG{n}{acc1}\PYG{o}{\PYGZgt{}} \PYG{o}{\PYGZlt{}}\PYG{n}{change1}\PYG{o}{\PYGZgt{}} \PYG{o}{\PYGZlt{}}\PYG{n}{mod1}\PYG{o}{\PYGZgt{}} \PYG{o}{\PYGZlt{}}\PYG{n}{nlink1}\PYG{o}{\PYGZgt{}}
\PYG{o}{.}\PYG{o}{.}\PYG{o}{.}
\PYG{o}{\PYGZlt{}}\PYG{n}{data\PYGZus{}n}\PYG{o}{\PYGZgt{}} \PYG{o}{\PYGZlt{}}\PYG{n}{uid\PYGZus{}n}\PYG{o}{\PYGZgt{}} \PYG{o}{\PYGZlt{}}\PYG{n}{gid\PYGZus{}n}\PYG{o}{\PYGZgt{}} \PYG{o}{\PYGZlt{}}\PYG{n}{dim\PYGZus{}n}\PYG{o}{\PYGZgt{}} \PYG{o}{\PYGZlt{}}\PYG{n}{perm\PYGZus{}n}\PYG{o}{\PYGZgt{}} \PYG{o}{\PYGZlt{}}\PYG{n}{acc\PYGZus{}n}\PYG{o}{\PYGZgt{}} \PYG{o}{\PYGZlt{}}\PYG{n}{change\PYGZus{}n}\PYG{o}{\PYGZgt{}} \PYG{o}{\PYGZlt{}}\PYG{n}{mod\PYGZus{}n}\PYG{o}{\PYGZgt{}} \PYG{o}{\PYGZlt{}}\PYG{n}{nlink\PYGZus{}n}\PYG{o}{\PYGZgt{}}
\PYG{c+c1}{\PYGZsh{}\PYGZsh{}\PYGZsh{}}
\PYG{o}{.}\PYG{o}{.}\PYG{o}{.}
\PYG{c+c1}{\PYGZsh{} \PYGZlt{}pathm\PYGZgt{}}
\PYG{o}{\PYGZlt{}}\PYG{n}{data1}\PYG{o}{\PYGZgt{}} \PYG{o}{\PYGZlt{}}\PYG{n}{uid1}\PYG{o}{\PYGZgt{}} \PYG{o}{\PYGZlt{}}\PYG{n}{gid1}\PYG{o}{\PYGZgt{}} \PYG{o}{\PYGZlt{}}\PYG{n}{dim1}\PYG{o}{\PYGZgt{}} \PYG{o}{\PYGZlt{}}\PYG{n}{perm1}\PYG{o}{\PYGZgt{}} \PYG{o}{\PYGZlt{}}\PYG{n}{acc1}\PYG{o}{\PYGZgt{}} \PYG{o}{\PYGZlt{}}\PYG{n}{change1}\PYG{o}{\PYGZgt{}} \PYG{o}{\PYGZlt{}}\PYG{n}{mod1}\PYG{o}{\PYGZgt{}} \PYG{o}{\PYGZlt{}}\PYG{n}{nlink1}\PYG{o}{\PYGZgt{}}
\PYG{o}{.}\PYG{o}{.}\PYG{o}{.}
\PYG{o}{\PYGZlt{}}\PYG{n}{data\PYGZus{}n}\PYG{o}{\PYGZgt{}} \PYG{o}{\PYGZlt{}}\PYG{n}{uid\PYGZus{}n}\PYG{o}{\PYGZgt{}} \PYG{o}{\PYGZlt{}}\PYG{n}{gid\PYGZus{}n}\PYG{o}{\PYGZgt{}} \PYG{o}{\PYGZlt{}}\PYG{n}{dim\PYGZus{}n}\PYG{o}{\PYGZgt{}} \PYG{o}{\PYGZlt{}}\PYG{n}{perm\PYGZus{}n}\PYG{o}{\PYGZgt{}} \PYG{o}{\PYGZlt{}}\PYG{n}{acc\PYGZus{}n}\PYG{o}{\PYGZgt{}} \PYG{o}{\PYGZlt{}}\PYG{n}{change\PYGZus{}n}\PYG{o}{\PYGZgt{}} \PYG{o}{\PYGZlt{}}\PYG{n}{mod\PYGZus{}n}\PYG{o}{\PYGZgt{}} \PYG{o}{\PYGZlt{}}\PYG{n}{nlink\PYGZus{}n}\PYG{o}{\PYGZgt{}}
\PYG{c+c1}{\PYGZsh{}\PYGZsh{}\PYGZsh{}}
\PYG{c+c1}{\PYGZsh{}\PYGZsh{}\PYGZsh{}}
\end{sphinxVerbatim}

Le informazioni associate al file/directory \sphinxcode{\sphinxupquote{\textless{}path\textgreater{}}} iniziano con la riga:

\begin{sphinxVerbatim}[commandchars=\\\{\}]
\PYG{c+c1}{\PYGZsh{} \PYGZlt{}path\PYGZgt{}}
\end{sphinxVerbatim}

Successivamente si trovano una sequenza di righe (una per ogni analisi svolta) della forma:

\begin{sphinxVerbatim}[commandchars=\\\{\}]
\PYG{o}{\PYGZlt{}}\PYG{n}{data}\PYG{o}{\PYGZgt{}} \PYG{o}{\PYGZlt{}}\PYG{n}{uid}\PYG{o}{\PYGZgt{}} \PYG{o}{\PYGZlt{}}\PYG{n}{gid}\PYG{o}{\PYGZgt{}} \PYG{o}{\PYGZlt{}}\PYG{n}{dim}\PYG{o}{\PYGZgt{}} \PYG{o}{\PYGZlt{}}\PYG{n}{perm}\PYG{o}{\PYGZgt{}} \PYG{o}{\PYGZlt{}}\PYG{n}{acc}\PYG{o}{\PYGZgt{}} \PYG{o}{\PYGZlt{}}\PYG{n}{change}\PYG{o}{\PYGZgt{}} \PYG{o}{\PYGZlt{}}\PYG{n}{mod}\PYG{o}{\PYGZgt{}} \PYG{o}{\PYGZlt{}}\PYG{n}{nlink}\PYG{o}{\PYGZgt{}}
\end{sphinxVerbatim}

Dove:

\begin{sphinxVerbatim}[commandchars=\\\{\}]
\PYG{o}{\PYGZlt{}}\PYG{n}{data}\PYG{o}{\PYGZgt{}} \PYG{n}{indica} \PYG{n}{ora}\PYG{o}{\PYGZhy{}}\PYG{n}{data} \PYG{o+ow}{in} \PYG{n}{cui} \PYG{n}{sono} \PYG{n}{recuperate} \PYG{n}{le} \PYG{n}{informazioni}\PYG{p}{;}
\PYG{o}{\PYGZlt{}}\PYG{n}{uid}\PYG{o}{\PYGZgt{}} \PYG{n}{è} \PYG{n}{l}\PYG{l+s+s1}{\PYGZsq{}}\PYG{l+s+s1}{id dell}\PYG{l+s+s1}{\PYGZsq{}}\PYG{n}{utente} \PYG{n}{proprietario} \PYG{k}{del} \PYG{n}{file}\PYG{p}{;}
\PYG{o}{\PYGZlt{}}\PYG{n}{gid}\PYG{o}{\PYGZgt{}} \PYG{n}{è} \PYG{n}{l}\PYG{l+s+s1}{\PYGZsq{}}\PYG{l+s+s1}{id del gruppo del file;}
\PYG{o}{\PYGZlt{}}\PYG{n}{perm}\PYG{o}{\PYGZgt{}} \PYG{n}{è} \PYG{n}{la} \PYG{n}{stringa} \PYG{n}{con} \PYG{n}{i} \PYG{n}{diritti} \PYG{n}{di} \PYG{n}{accesso} \PYG{n}{al} \PYG{n}{file}\PYG{p}{;}
\PYG{o}{\PYGZlt{}}\PYG{n}{acc}\PYG{o}{\PYGZgt{}} \PYG{n}{data} \PYG{n}{dell}\PYG{l+s+s1}{\PYGZsq{}}\PYG{l+s+s1}{ultimo accesso;}
\PYG{o}{\PYGZlt{}}\PYG{n}{change}\PYG{o}{\PYGZgt{}} \PYG{n}{data} \PYG{n}{dell}\PYG{l+s+s1}{\PYGZsq{}}\PYG{l+s+s1}{ultimo cambiamento;}
\PYG{o}{\PYGZlt{}}\PYG{n}{mod}\PYG{o}{\PYGZgt{}} \PYG{n}{data} \PYG{n}{dell}\PYG{l+s+s1}{\PYGZsq{}}\PYG{l+s+s1}{ultima modifica dei permessi;}
\PYG{o}{\PYGZlt{}}\PYG{n}{nlink}\PYG{o}{\PYGZgt{}} \PYG{n}{numero} \PYG{n}{di} \PYG{n}{link} \PYG{n}{verso} \PYG{n}{il} \PYG{n}{file}\PYG{o}{.}
\end{sphinxVerbatim}

Le informazioni terminano con la riga:

\begin{sphinxVerbatim}[commandchars=\\\{\}]
\PYG{c+c1}{\PYGZsh{}\PYGZsh{}\PYGZsh{}}
\end{sphinxVerbatim}

Il file di output termina con una riga:

\begin{sphinxVerbatim}[commandchars=\\\{\}]
\PYG{c+c1}{\PYGZsh{}\PYGZsh{}\PYGZsh{}}
\end{sphinxVerbatim}


\chapter{Realizzazione del progetto}
\label{\detokenize{project:realizzazione-del-progetto}}\label{\detokenize{project::doc}}
La realizzazione del codice prodotto ha seguito uno standard \sphinxstylestrong{preciso} ed \sphinxstylestrong{efficente} che ha reso lo sviluppo di quest’ultimo \sphinxstylestrong{flessibile} ed \sphinxstylestrong{elegante} ai fini
di aver un’utility di sistema \sphinxstyleemphasis{altamente performante} grazie alle potenzialità offerte dal \sphinxcode{\sphinxupquote{C}} stesso.


\section{Struttura e architettura del codice sviluppato}
\label{\detokenize{project:struttura-e-architettura-del-codice-sviluppato}}
La struttura del progetto si articola fondamentalmente su 5 file sorgente di estensione \sphinxcode{\sphinxupquote{.c}}, a cui seguono altrettanti file di estensione \sphinxcode{\sphinxupquote{.h}}, in cui vengono dichiari i metodi da \sphinxstyleemphasis{estendere}:
\begin{itemize}
\item {} 
\sphinxcode{\sphinxupquote{main.c}}: contiene il codice di avvio del progetto. Consente il \sphinxstylestrong{parse} delle opzioni e la corretta apertura dei file di \sphinxcode{\sphinxupquote{input}} e dei file di \sphinxcode{\sphinxupquote{output}}.

\item {} 
\sphinxcode{\sphinxupquote{datastructure.c}}: contiene il codice necessario alla gestione della \sphinxstylestrong{struttura dati} impiegata nel progetto per la collezione dei dati relativi ai \sphinxstylestrong{file monitorati}.

\item {} 
\sphinxcode{\sphinxupquote{scan.c}}: contiene il codice necessario all’inizializzazione della \sphinxstylestrong{struttura} dati tramite la lettura delle informazioni specificate tramite i file di input e output.

\item {} 
\sphinxcode{\sphinxupquote{inputscan.c}}: contiene il codice finalizzato all’analisi del file di \sphinxcode{\sphinxupquote{input}} e dei file i cui \sphinxstylestrong{pathname} sono specificati nel file di \sphinxcode{\sphinxupquote{input}}. Aggiorna di conseguenza la \sphinxstylestrong{struttura dati} con le informazioni relative ai \sphinxstylestrong{file monitorati}.

\item {} 
\sphinxcode{\sphinxupquote{output.c}}: contiene il codice finalizzato all’analisi del file di \sphinxcode{\sphinxupquote{output}} per poter inserire le informazioni contenute al suo interno nella \sphinxstylestrong{struttura dati} specificata in precedenza.

\end{itemize}

Per l’analisi dei singoli file sorgenti si rimanda alle sezioni dedicate alla spiegazione dei singoli metodi specificati al loro interno.

Le librerie impiegate all’interno del codice sono:
\begin{itemize}
\item {} 
\sphinxcode{\sphinxupquote{Libreria standard di C}}: sono stati usati gli header \sphinxcode{\sphinxupquote{\textless{}stdio.h\textgreater{}}}, \sphinxcode{\sphinxupquote{\textless{}string.h\textgreater{}}}, \sphinxcode{\sphinxupquote{\textless{}stdlib.h\textgreater{}}}, \sphinxcode{\sphinxupquote{\textless{}errno.h\textgreater{}}}, \sphinxcode{\sphinxupquote{\textless{}time.h\textgreater{}}}.

\item {} 
\sphinxcode{\sphinxupquote{Libreria POSIX C}}: sono stati usati gli header \sphinxcode{\sphinxupquote{\textless{}unistd.h\textgreater{}}}, \sphinxcode{\sphinxupquote{\textless{}sys/stat.h\textgreater{}}}, \sphinxcode{\sphinxupquote{\textless{}limits.h\textgreater{}}}, \sphinxcode{\sphinxupquote{\textless{}pwd.h\textgreater{}}}, \sphinxcode{\sphinxupquote{\textless{}grp.h\textgreater{}}}, \sphinxcode{\sphinxupquote{\textless{}dirent.h\textgreater{}}}, \sphinxcode{\sphinxupquote{\textless{}getopt.h\textgreater{}}}.

\end{itemize}


\section{Struttura dati implementata}
\label{\detokenize{project:struttura-dati-implementata}}
\noindent\sphinxincludegraphics{{ssdraw}.png}

La struttura dati alla base del funzionamento del progetto è stata definita tramite gli struct \sphinxcode{\sphinxupquote{pathentry}} e \sphinxcode{\sphinxupquote{analisisentry}}, entrambi definiti in \sphinxcode{\sphinxupquote{datastructure.h}} e associati
ai tipi \sphinxcode{\sphinxupquote{PathEntry}} e \sphinxcode{\sphinxupquote{AnalisisEntry}} definiti tramite il costrutto \sphinxcode{\sphinxupquote{typedef}}.
Lo struct \sphinxcode{\sphinxupquote{pathentry}} è indirizzato alla definizione di una lista in cui ogni elemento contiene una stringa, associata ad un pathname, il puntatore ad un elemento \sphinxcode{\sphinxupquote{AnalisisEntry}}
ed il puntatore ad un elemento \sphinxcode{\sphinxupquote{PathEntry}}, l’elemento successivo all’interno della lista. Analogamente lo struct \sphinxcode{\sphinxupquote{analisisentry}} è puntato alla definizione di una lista i cui elementi
contentono una stringa associata alle informazioni relative all’analisi di un file ed il puntatore ad un elemento \sphinxcode{\sphinxupquote{AnalisisEntry}}, elemento successivo nella lista.

Elementi \sphinxcode{\sphinxupquote{PathEntry}} e \sphinxcode{\sphinxupquote{AnalisisEntry}} che non contengono informazioni sono associati al puntatore \sphinxstylestrong{NULL}.
Le funzionalità incluse all’interno del file \sphinxcode{\sphinxupquote{datastructure.h}} permettono di ottenere puntatori ad elementi vuoti di \sphinxcode{\sphinxupquote{PathEntry}} e \sphinxcode{\sphinxupquote{AnalisisEntry}}, di aggiungere ad una lista di \sphinxcode{\sphinxupquote{PathEntry}}
nuovi elementi tramite il passaggio di stringhe contenenti pathname e le informazioni derivate dall’analisi dei file associati a tali pathname, di verificare che una lista di \sphinxcode{\sphinxupquote{PathEntry}}
o di \sphinxcode{\sphinxupquote{AnalisisEntry}} risulti vuota, di ottenere gli elementi successivi all’interno di una lista \sphinxcode{\sphinxupquote{PathEntry}} o \sphinxcode{\sphinxupquote{AnalisisEntry}} dato il puntatore ad un elemento delle due liste e di ottenere
il riferimento al primo elemento della lista di \sphinxcode{\sphinxupquote{AnalisisEntry}} associata ad un dato elemento di una lista di \sphinxcode{\sphinxupquote{PathEntry}}.
Per ulteriori informazioni si rimanda alla sezione dedicata a \sphinxcode{\sphinxupquote{datastructure.c}}, in cui sono presenti le definizioni dei metodi dichiarati in \sphinxcode{\sphinxupquote{datastructure.h}} e in \sphinxcode{\sphinxupquote{datastructure.c}}.

Essendo la struttura dati alla base del progetto basata su due implementazioni di una lista la sua complessità risulta essere:
\begin{itemize}
\item {} 
\sphinxcode{\sphinxupquote{O(P)}} quando è necessario aggiungere o recuperare le informazioni da uno specifico elemento \sphinxcode{\sphinxupquote{PathEntry}}.

\item {} 
\sphinxcode{\sphinxupquote{O(P x A)}} quando è necessario aggiungere o recuperare le informazioni da uno specifico elemento \sphinxcode{\sphinxupquote{AnalisisEntry}}.

\end{itemize}

Dove \sphinxcode{\sphinxupquote{P}} rappresenta il numero di elementi \sphinxcode{\sphinxupquote{PathEntry}} presenti nella struttura dati e \sphinxcode{\sphinxupquote{A}} il numero massimo di elementi \sphinxcode{\sphinxupquote{AnalisisEntry}} associati ad un \sphinxcode{\sphinxupquote{PathEntry}}.


\section{Implementazione delle funzionalità richieste}
\label{\detokenize{project:implementazione-delle-funzionalita-richieste}}
L’esecuzione del programma porta ad un aggiornamento complessivo delle informazioni contenute all’interno del file output, aggiungendo \sphinxcode{\sphinxupquote{pathname}} se non già presenti e
analisi delle informazioni relative ai file referenziati dai \sphinxcode{\sphinxupquote{pathname}} presenti e aggiunti. Al termine dell’esecuzione il file di \sphinxcode{\sphinxupquote{output}} risulterà pertanto aggiornato e
non interamente sovrascritto. I pathname aggiunti al file di output gestito dal programma sono tutti assoluti per permettere la portabilità di tale file.

Per la gestione delle decisioni dell’utente circa l’uso di file di \sphinxcode{\sphinxupquote{input}} e di \sphinxcode{\sphinxupquote{output}} che non siano quelli di default si è deciso di assicurare le funzionalità del programma
solo in presenza o in assenza dei \sphinxcode{\sphinxupquote{pathname}} associati ad \sphinxstylestrong{entrambi} i file. L’inclusione di un singolo \sphinxcode{\sphinxupquote{pathname}} all’interno della sinossi di avvio del programma porterà
all’avvio del programma con l’impiego dei file di \sphinxcode{\sphinxupquote{input}} e di \sphinxcode{\sphinxupquote{output}} di default.

L’uso dell’opzione \sphinxcode{\sphinxupquote{-v/-{-}verbose}} porterà alla stampa sullo \sphinxstylestrong{standard output} di informazioni circa i file analizzati, quali il loro \sphinxcode{\sphinxupquote{pathname}} relativo (che diventa assoluto nel caso il file
analizzato sia un file referenziato da un link), l’eventuale natura di link o di directory e la corretta riuscita dell’operazione di analisi delle informazioni.
L’implementazione delle opzioni \sphinxcode{\sphinxupquote{-s/-{-}stat}} e \sphinxcode{\sphinxupquote{-r/-{-}report}} risulta la medesima e consiste nella stampa sullo \sphinxstylestrong{standard output} delle informazioni richieste alla fine dell’elaborazione generale.
L’uso dell’opzione \sphinxcode{\sphinxupquote{-h/-{-}history}}, seguita dal pathname del file di cui si vuole ottenere la cronologia, porterà alla stampa sullo \sphinxstylestrong{standard output} delle informazioni relative al file associato a tale pathname.
Se il pathname non è presente all’interno del file di output gestito dal programma verrà effettuata una notifica sullo \sphinxstylestrong{standard output} di tale mancanza. Non sono ovviamente incluse le informazioni
aggiunte tramite l’esecuzione del programma in corso.
L’implementazione di \sphinxcode{\sphinxupquote{-u/-{-}user}} e \sphinxcode{\sphinxupquote{-g/-{-}group}} consiste in un \sphinxstylestrong{filtro} effettivo su quelli che sono i file da monitorare e di cui aggiungere informazioni nel file di \sphinxcode{\sphinxupquote{output}}.
L’inclusione di un file all’interno dell’operazione di analisi effettuata dal programma, \sphinxstyleemphasis{in presenza di tali opzioni}, porta alla stampa sullo \sphinxstylestrong{standard output} del pathname assoluto del file incluso nell’analisi
e delle relative informazioni collezionate.
Il medesimo discorso si applica anche all’implementazione di \sphinxcode{\sphinxupquote{-l/-{-}length}}.
Infine, per l’implementazione di \sphinxcode{\sphinxupquote{-{-}noscan}} si è deciso di effettuare comunque l’operazione di popolamento della struttura dati con le informazioni derivate dal file di output evitando ogni tipo
di operazione di analisi su ulteriori file, come quelli specificati dai pathname presenti nel file di input, e portando alla stampa sullo \sphinxstylestrong{standard output} delle informazioni presenti all’interno della struttura
dati al termine delle operazioni del programma.


\section{Makefile}
\label{\detokenize{project:makefile}}\begin{quote}

Il make è un’utility, sviluppata sui sistemi operativi della famiglia UNIX, ma disponibile su un’ampia gamma di sistemi, che automatizza
il processo di creazione di file che dipendono da altri file, risolvendo le dipendenze e invocando programmi esterni per il lavoro necessario.
\end{quote}

Tale utility nel nostro caso è stata utilizzata per la compilazione di \sphinxstylestrong{codice sorgente} in \sphinxstylestrong{codice oggetto}, unendo e poi linkando il codice oggetto
in un programma eseguibile chiamato \sphinxcode{\sphinxupquote{filestat}}.

Essa usa file chiamati \sphinxcode{\sphinxupquote{makefile}} per determinare il grado delle dipendenze per un particolare output, e gli script
necessari per la compilazione da passare alla shell.

I \sphinxstyleemphasis{task} che mette a disposizione sono i seguenti:
\begin{itemize}
\item {} 
\sphinxcode{\sphinxupquote{make filestat}}: converte il codice sorgente realizzato, \sphinxstyleemphasis{con le librerie a lui annesse}, in un codice oggetto eseguibile lanciando il comando \sphinxcode{\sphinxupquote{./filestat}}

\item {} 
\sphinxcode{\sphinxupquote{make clean}}: elimina il contenuto delle directory indicate al suo interno per ottenere sempre un ambiente di lavoro pulito e privo di file obsoleti

\item {} 
\sphinxcode{\sphinxupquote{make test}}: generazione della cartella principale \sphinxcode{\sphinxupquote{folder\_testing}} in grado di dare all’utente \sphinxstylestrong{la possibilità} di testare il corretto funzionamento dell’utility \sphinxcode{\sphinxupquote{filestat}}

\end{itemize}


\section{Test relativi al corretto funzionamento}
\label{\detokenize{project:test-relativi-al-corretto-funzionamento}}
Per avere una stima rispetto al corretto funzionamento del codice sono stati effettuati, in primo luogo,
dei test molto \sphinxstyleemphasis{spartani} mediante i comandi \sphinxcode{\sphinxupquote{ls -l}}, \sphinxcode{\sphinxupquote{du -sh file\_path}} e utilizzando l”\sphinxstyleemphasis{explorer} di sistema fornito dall’OS.

Quest’ultimi ci restituivano infatti le informazioni \sphinxstylestrong{corrette} rispetto ai dati che analizzavamo, e in
maniera banale, li confrontavamo con quelli che l’utility produceva. Una volta confermato il corretto funzionamento
dell’utility si è deciso quindi di produrre una script per \sphinxcode{\sphinxupquote{bash}} che fosse in grado di generare in maniera del tutto
casuale file, link e directory, che a loro volta contenevano altrettanti elementi, per testare in maniera definitiva
l’utility stessa e dimostrare in maniera oggettiva il suo funzionamento.

Da questa premessa nasce infatti \sphinxcode{\sphinxupquote{folder\_testing.sh}}.

Lo script in questione, disponibile all’interno della main direcotry del progetto, attinge a risorse di sistema localizzate
in \sphinxcode{\sphinxupquote{/dev/urandom}} per produrre dei contenuti di natura \sphinxstylestrong{random} relativi ai nomi dei file e delle directory e
per popolare il loro contenuto.

L’esecuzione di tale script quindi genera una nuova direcotry \sphinxcode{\sphinxupquote{folder\_testing}} al cui interno sarà possibile
trovare i file - \sphinxstyleemphasis{e le direcotry} - nati da tale generazione.

Per avviare tale processo sarà necessario lanciare il comando:

\begin{sphinxVerbatim}[commandchars=\\\{\}]
\PYG{n}{make} \PYG{n}{test}
\end{sphinxVerbatim}

Infatti all’interno del \sphinxstylestrong{Makefile} di cui si è parlato nella sezione relativa a codesto argomento è possibile reperire tale informazione.

È interessante poi vedere come l’implementazione e il lancio di tale script produca subito un risultato tangibile che attesti il numero di file
e directory generate, così come il numero di link presenti e in particolar modo la somma complessiva del peso di tali file.

Di seguito è possibile apprezzare la bontà e la comodità di tale \sphinxcode{\sphinxupquote{script}}:

\begin{sphinxVerbatim}[commandchars=\\\{\}]
./folder\PYGZus{}testing
├── [       4096]  ICcJo
│   ├── [       4096]  2ymehX
│   │   └── [       4096]  QP
│   │       ├── [       4026]  8qR1g46s.bin
│   │       ├── [      16086]  bdkx0.bin
│   │       └── [       3837]  ezK6fiUW3dAR.bin
│   ├── [       4096]  87
│   │   ├── [       4096]  QP
│   │   ├── [       2109]  tdzY.bin
│   │   └── [      16310]  YpdiX.bin
│   └── [       4096]  JmqQ
│       └── [       4096]  QP
│           ├── [       3652]  Prdg0.bin
│           └── [        861]  yMutQoBsI.bin
├── [       4096]  Si1
│   ├── [       4096]  Aw
│   │   ├── [       4096]  0BxaN
│   │   │   ├── [       6707]  nq.bin
│   │   │   └── [      11253]  pWIsm.bin
│   │   ├── [       4096]  aG
│   │   │   └── [       9555]  ZbioiWDOw.bin
│   │   ├── [       4096]  h3n1
│   │   │   └── [       2694]  8m.bin
│   │   ├── [       4096]  rFes
│   │   │   ├── [      19273]  RRP.bin
│   │   │   └── [       8035]  YpaCzp.bin
│   │   └── [        677]  QT0Dwlb3.bin
│   ├── [       4096]  fkXD
│   │   ├── [       4096]  0BxaN
│   │   │   └── [       2503]  aIqEChA.bin
│   │   ├── [       4096]  aG
│   │   │   └── [       1507]  G5MF0.bin
│   │   ├── [       4096]  h3n1
│   │   │   └── [       3017]  1YF3kYej9P.bin
│   │   ├── [       4096]  rFes
│   │   ├── [       6573]  07b.bin
│   │   └── [       3764]  qBbF.bin
│   ├── [       4096]  l jPwu3
│   │   ├── [       4096]  0BxaN
│   │   │   ├── [       9030]  h5FDXIsn.bin
│   │   │   └── [       2658]  XomM4.bin
│   │   ├── [       4096]  aG
│   │   │   ├── [       2980]  QUWYJY.bin
│   │   │   └── [      10209]  RgziPE7jj.bin
│   │   ├── [       4096]  h3n1
│   │   │   └── [       3405]  xKFZ6j.bin
│   │   ├── [       4096]  rFes
│   │   │   ├── [      11185]  17s.bin
│   │   │   └── [       6144]  uJW1U.bin
│   │   ├── [       8663]  Df8rs.bin
│   │   └── [      12494]  Vqo8R.bin
│   ├── [       4096]  M F
│   │   ├── [       4096]  0BxaN
│   │   │   └── [      14186]  rCt35X.bin
│   │   ├── [       4096]  aG
│   │   │   └── [        359]  j0Bl7jx.bin
│   │   ├── [       4096]  h3n1
│   │   │   ├── [      17973]  SE4XX8ExlK.bin
│   │   │   └── [       4964]  sfiq0Q.bin
│   │   └── [       4096]  rFes
│   │       └── [      13436]  P9hWuujV.bin
│   ├── [       1663]  G6G0n8W7.bin
│   └── [      10078]  RjpHS.bin
├── [          7]  link\PYGZus{}0 \PYGZhy{}\PYGZgt{} Mru.bin
└── [       5795]  Mru.bin

28 directories, 37 files
Dimensione totale dei file: 376452      ./folder\PYGZus{}testing
\end{sphinxVerbatim}

Dopo aver lanciato tale comando infatti basterà modificare il percorso da analizzare all’interno del file di input fornito
per poi confrontarle con quelle restituire dall’utility prodotta.



\renewcommand{\indexname}{Indice}
\printindex
\end{document}